%%%%%%%%%%%%%%%%%%%%%%%%%%%%%%%%%%%%%%%%%
% Medium Length Professional CV
% LaTeX Template
% Version 2.0 (8/5/13)
%
% This template has been downloaded from:
% http://www.LaTeXTemplates.com
%
% Original author:
% Trey Hunner (http://www.treyhunner.com/)
%
% Important note:
% This template requires the resume.cls file to be in the same directory as the
% .tex file. The resume.cls file provides the resume style used for structuring the
% document.
%
%%%%%%%%%%%%%%%%%%%%%%%%%%%%%%%%%%%%%%%%%

%----------------------------------------------------------------------------------------
%	PACKAGES AND OTHER DOCUMENT CONFIGURATIONS
%----------------------------------------------------------------------------------------

\documentclass{resume} % Use the custom resume.cls style

\usepackage[left=0.2in,top=0.2in,right=0.2in,bottom=0.25in]{geometry} % Document margins

\name{Yunzhong He} % Your name
\address{3757 Kelton Ave, 106 \\ Los Angeles, CA 90034} % Your address
\address{(310)~$\cdot$~923~$\cdot$~1381 \\ hyz331@gmail.com} % Your phone number and email

\begin{document}

%----------------------------------------------------------------------------------------
%	EDUCATION SECTION
%----------------------------------------------------------------------------------------

\begin{rSection}{Education}

{\bf University of California, Los Angeles} \hfill {\em September 2015 - December 2016} \\ 
M.S. in Computer Science, focus on Artificial Intelligence and Computer Vision \\
Currently working at VCLA (Center for Vision, Cognition, Learning and Autonomy) with Prof. Song-Chun Zhu \\
\newline
{\bf University of California, Los Angeles} \hfill {\em September 2011 - April 2015} \\
B.S. in Computer Science with a focus on Mathematics \\
Graduated with Cum Laude, Member of Upsilon Pi Epsilon (Computer Science Honor Society)
\end{rSection}

%----------------------------------------------------------------------------------------
%	WORK EXPERIENCE SECTION
%----------------------------------------------------------------------------------------

\begin{rSection}{Work Experience}

\begin{rSubsection}{Electronic Arts}{June 2014 - today}{Data Engineer Intern at EA Digital Platform - Data Team}{Redwood City, CA}
\item Work on text mining of some game data, build a statistical model to summerize valuable information
\item Integrate the model into Hadoop/Spark, define API and create visualizations of mining results
\end{rSubsection}

%------------------------------------------------


\begin{rSubsection}{Amazon Web Services}{May 2015 - September 2015}{Software Development Engineer Intern at AWS CloudDrive - Content Processing Team}{Seattle, WA}
\item Design and implement services that live on AWS to extract valuable information from uploaded files, perform data normalization and define index to make them searchable by users
\item Analyze statistics and perform optimizations to scale the services to production
\end{rSubsection}

%------------------------------------------------

\begin{rSubsection}{Qualcomm}{June 2014 - September 2014}{Software Engineer Intern at APT Linux Team}{San Diego, CA}
\item Work on various features of a test execution engine and a test report management system for Snapdragon processors, build a testing script monitor using Spring MVC and Hibernate
\item Work on an Android battery stress tester
\end{rSubsection}

%------------------------------------------------

\begin{rSubsection}{Silvus Technologies}{June 2013 - December 2013}{Embedded Software Engineer Intern}{Los Angeles, CA}
\item Work on Linux custimization and networking softwares of cutting-edge MIMO radios
\item Design and implement a user authentication and data encryption module using C and shell scripts
\end{rSubsection}

\end{rSection}

%----------------------------------------------------------------------------------------
%	RESEARCH PROJECTS
%----------------------------------------------------------------------------------------

\begin{rSection}{Research and Teaching}
\textbf{Robot Learning from Human Demonstration and Dialog}\\
A unified framework for robots to learn arbitrary tasks from video demonstrations and dialogue. The knowledge representation is based on a stochastic grammar model, and the system involves feature extraction using auto-encoders, and causality learning from observations. Relevant papers submitted to \textbf{NIPS 2016} (first author) and \textbf{ACL 2016}\\\\
\textbf{Teaching Assistant at UCLA}\\  
Formal Languages and Automata Theory (Fall 2015, Spring 2016)\\
Mathematical Methods and Models for Computer Science (Winter 2016)
\end{rSection}


%----------------------------------------------------------------------------------------
%	TECHNICAL SKILLS SECTION
%----------------------------------------------------------------------------------------

\begin{rSection}{TECHNICAL SKILLS}

\begin{tabular}{ @{} >{\bfseries}l @{\hspace{6ex}} l }
Computer Languages & C/C++, Java, Python, Lisp, PHP, JavaScript/Jquery, HTML/CSS, Matlab, Assembly \\
Machine Learning & Familiar with machine learning algorithms and tools, focus on graphical models \\
Big Data Technologies & Familiar with AWS, Hadoop and Spark \\
Web Development & Java Spring, CodeIgniter, CakePHP, Joomla, EmberJS, SQL, NoSQL, etc. \\
Agile Development & Familiar with agile development process and unit testing tools (JUnit, Mockito, etc.) \\
\end{tabular}
\end {rSection}

%----------------------------------------------------------------------------------------
%	OTHER SECTION
%----------------------------------------------------------------------------------------

% \begin{rSection}{Other}
% Member of Upsilon Pi Epsilon (Computer Science Honor Society) \\
% Worked on a paper in Bioinformatics with Prof Kirk Lohmueller 
% \end {rSection}


\end{document}
